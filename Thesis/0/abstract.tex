
% Thesis Abstract -----------------------------------------------------


%\begin{abstractslong}    %uncommenting this line, gives a different abstract heading
\begin{abstracts}
Computer that coordinate or interact with human 
musicians exist in many forms
and is used in music performance for many years, from  
computer instruments to computer accompaniment, from compostion system 
to conducting system.
Human Computer Music Performance (HCMP) 
project aims to create a more autonomous ``artificial performer''    
that does not require human interference. To develop a practice of
HCMP, we need to develop a HCMP Player which is able to flexbily 
change the tempo and communicate with other components of HCMP components. 

The HCMP Player is designed and implemented with 2 goals in mind. The first
goal is to design a midi player which has a both time and space efficient tempo 
changing algorithm, it need to deal with frequent tempo changing during 
performance within bounded time. The second goal is to clearly define a set of 
programming interface in the HCMP player, which can be easily extended 
and used to cooperate and coordinate with other components of HCMP project.
In this paper, I describe the whole development process of HCMP player, 
from design logic of code structure to general software architecture, 
from implementation to complete software testing plan. Challenges, Problems 
during development and Key design logic 
behind implementation is also discussed and explained in the paper.

\end{abstracts}
%\end{abstractlongs}
