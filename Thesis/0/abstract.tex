% Thesis Abstract -----------------------------------------------------
\begin{abstracts}
From computer instruments to computer accompaniment, from composition
systems to conducting systems, musicians have been interacting with
computers for many years. The Human Computer Music Performance (HCMP)
project aims to create a more autonomous ``artificial  performer'' that does
not require human input or adjustment. The heart of the HCMP project is the
development of an HCMP Player which is able to flexibly change the tempo and
communicate with other HCMP components.

The HCMP Player is designed and implemented with two goals in mind. The
first goal is to design a MIDI player which has an efficient time and space
tempo changing algorithm. The algorithm must process frequent tempo changes
during a performance and within bounded time periods. The second goal is to
clearly define a set of programming interfaces in the HCMP player, which can
be easily extended and used to coordinate the components of the HCMP
project. In this paper, I describe the development process of the HCMP
player, from design logic and code structure to general software
architecture, from implementation to a complete software testing plan.
Challenges, problems during development and key design logic behind the
implementation are also covered.
\end{abstracts}
