% this file is called up by thesis.tex
% content in this file will be fed into the main document

%: ----------------------- name of chapter  -------------------------
\chapter{Use Case} % top level followed by section, subsection


%: ----------------------- paths to graphics ------------------------

% change according to folder and file names
\ifpdf
    \graphicspath{{X/figures/PNG/}{X/figures/PDF/}{X/figures/}}
\else
    \graphicspath{{X/figures/EPS/}{X/figures/}}
\fi

%: ----------------------- contents from here ------------------------


Based on the project design, the evaluation of the project mainly contains two parts. To reach the
successful criteria, the project must satisfy both functional and performance requirement.

\section{Functional Evaluation}
\subsection{stand-alone mode}
\begin{enumerate}
  \item User should be able to load and play normal MIDI files.
  \item User should be able to play, pause, stop while playing MIDI file.  
  \item User should be able to freely adjust tempo.  
  \item Data visualization component should correctly show the MIDI note 
        and synchronize with audio stream while playing the MIDI file.
  \item Each track component should match according color on data visualization component.
  \item Each track component should be able to mute, solo and adjust 
        volume for according track.  
  \item The player should contain a simple MIDI library to manage MIDI files. 
        The library should automatically remember frequently used MIDI files.
  \item The player should provide a complete configuraion interface for the 
        user, all user's setting should be saved automatically.
\end{enumerate}

\subsection{network mode}
\begin{enumerate}
  \item Each instance of the HCMP player runs in its own process space, and should 
        be launched separately by user.
        After launching, the HCMP player should automatically connect to a conductor 
        based on its configuration.
  \item The Conductor communicate with the HCMP player through ZMQ protocol 
        \cite{ZMQ}, adapting the publisher-subscriber design pattern. The 
        Conductor will be ``publisher'' and each MIDI player will be ``subscriber''. 
        In this process, The MIDI player first connect to the Conductor and 
        registered as subscriber, the Conductor will inform all the 
        subscribers and invoke according callback function once an event happens.
  \item Multiple instances of MIDI player should be able to connect to one conductor. 
  \item After establishing connection with the Conductor, the MIDI player 
        should be able to receive tempo informaiton from the Conductor and 
        synchronize with it.     
  \item The Conductor should be able to ask certain MIDI player to load specified configuration file or it could directly 
        send configuration information through network to MIDI player.
  \item 
    The HCMP player is compatible with ZeroMQ \cite{zeromq} api, and 4 core 
    functions will be like
    \begin{itemize}
      \item play - ask a player to start playing 
      \item stop - stop a player from playing 
      \item update time-map - send new time-map to the player 
      \item set position - set position to certain beat 
    \end{itemize}
\end{enumerate}

%\section{Performance Evaluation}
\begin{enumerate}
  \item The HCMP player GUI part should always response to user input without 
        any delay, and the GUI thread should not hang there.
  \item The HCMP player should response to the Conductor without significant delay.
  \item The HCMP player should use no more than 100MB memory.
  \item The HCMP player should use reasonable amount of CPU while playing MIDI file.
  \item Hopefully, the HCMP player shoud not crash without any reason. 
\end{enumerate}
