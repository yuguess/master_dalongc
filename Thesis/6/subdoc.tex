% this file is called up by thesis.tex
% content in this file will be fed into the main document

%: ----------------------- name of chapter  -------------------------
\chapter{Implementation} % top level followed by section, subsection


%: ----------------------- paths to graphics ------------------------

% change according to folder and file names
\ifpdf
    \graphicspath{{X/figures/PNG/}{X/figures/PDF/}{X/figures/}}
\else
    \graphicspath{{X/figures/EPS/}{X/figures/}}
\fi

\section{Overview}
Chapter 5 mainly describe the implemenation detail of HCMP Player. Some 
pseudocode snippets are given to illustrate the relationship between 
different components. 
Focus will be on discussing some implementation related issues. 

\section{Development Environment}

The HCMP Player is developed using script language 
serpent\cite{Serpent}, serpent is similar to python, which has a simple, 
minimal syntax , dynamic typing and support for object-oriented programming. 
The best part of serpent is 
that it has many convenient builtin functions for midi messages processing, 
which greatly facilitate developing music-related applications. 
As for the GUI part, wxWidgets \cite{wxWidget} library provides one of best 
solution, 
some core functions and objects has been integrated into the serpent library. 
The zeromq library is used to develop network related functions, it integrate 
with common network commmunication functions. The whole project is developed 
in Linux 32-bit environment, serpent provide a cross-platform solution, 
so the HCMP Player should also be run in Windows or Mac platform.

\section{GUI Implementation}
Table 5.1 illustrate the general code structure of GUI component, 
it firstly parse the configuration file, if there is no configuration file 
in current working directory, then the HCMP Player will create a new one.
The configuration is stored 
at current working directory with name ``\texttt{.player\_config}'', and 
contains all the HCMP Player  
related settings, for example, the default remote server address and 
port number, recently openned file path etc. Configuration is written with plain 
ASCII text, user could easily modify the configuration. GUI component has 2 
operations, new and save to create a new configuration file and save current 
configurations to the file.

After the configuration parsing and initialization job complete, the 
the GUI begins to create each subcomponents. Each subcomponent has defined its 
own object construction function. Note that,  to make data accessing easier 
between different components, we pass in ``this'' (object itself) to each  
subcomponents.
\begin{table}[htdp]
\centering
\begin{lstlisting}

def GUI {
  data_init() //parse configuration file
  create_menu_panel(this) // initialize all the sub-components
  create_track_panel(this)
  create_data_panel(this)
  create_library_panel(this)
}

\end{lstlisting}
\caption[GUI Pseudocode Snippet]{GUI Pseudocode Snippet}
\end{table}

\subsection{Binding Method}
Table 5.2 is a serpent code snippet, which shows a simple way for GUI object 
to bind with a method, we bind play button a method, which send a play 
request to player engine, note that
the argument of binding method is a string with method name and parameters
separated by \texttt{;}. The binding function's type is predefined inside 
serpent, which has 4 prefined arguments. The first argument is object itself, 
the second arugment is event type, thrid and fourth argument is event position,
note that not all of the event argument is valid, some events provide 4 arguments, 
some events may only need 2 arguments. In serpent, all the object event is 
dispatched through the same user defined fucntion, which give user full power 
to develop its own customized handlers.

\begin{table}[htdp]
\centering
\begin{lstlisting}

Class PlayBtn(button)
  def init()
    super.init()
    this.method = 'send_play'

  def send_play(obj, event, x, y)
    send_player_engine('play;void')

\end{lstlisting}
\caption[Binding Method]{Binding Method}
\end{table}

\subsection{Drawing Canvas}
 
Canvas object define an specified area to be drawable. In serpent, the canvas 
use the wxWidget canvas object, the canvas object has a full-size bitmap, 
the drawing method
first draw on the bitmap, then the whole bitmap is copied to canvas. The code 
has been special optimized to deal with bitmap-copy, so the efficiency is not 
big issue. In serpent, a class need to fulfill two conditions in order 
to be drawable. First condition it
the class need to be a subclass of canvas class, second condition is the class  
need to define its own paint method. Table 5.3 illustrate the user defined 
\texttt{Keyboard} class, which is to draw a keyboard on canvas.

In HCMP Player, there are two classes which adapts canvas class, \texttt{Keyboard} 
and \texttt{MidiNotes} class, the former draws a 108-key keyboard, the latter
draws all midi notes into canvas. Each class has a buffer to store all the objects
need to be drawn. The play engine periodically send ``synchronization'' signal,  
everytime, GUI receive the signal, it will redraw the whole area.

\begin{table}[htdp]
\centering

\begin{lstlisting}
class Keyboard(Canvas)
    def init(parent, x, y, w, h, b, c)
      super.init(parent, x, y, w, h)

    def paint(x)
      draw_octave()
\end{lstlisting}
\caption[Serpent Canvas]{Serpent Canvas}
\end{table}

\section{Player Engine Implementation}

\subsection{Create New Thread}
Player engine runs inside a separate thread space, when GUI thread complete all
the subcomponets' initialization work, it will create a new thread, the newly
spawned thread will execute the player engine code. In table 5.4, we use the   
\texttt{thread\_create} builtin function of serpent standard library to create 
new thread, it has 3 arguments, 
the first argument is to set a period by which newly spawned will wake up and call 
callback-style function, the mechanism can be used to set a message channel between 
two threads for communication purposes. The second argument specify 
which script file to execute for the new thread, the third argument specify the 
memory size for the new thread.

\begin{table}[htdp]
\centering

\begin{lstlisting}

g_object = GUI() //complete constructing GUI object
thread_create(2, "player_engine.srp", 0) // create new thread

\end{lstlisting}

\caption[Create New Thread]{Create New Thread}
\end{table}

\subsection{Thread Communication}

Serpent has a very primitive multi-thread interface, the thread concept is 
different from how thread works in C, though serpent 
thread exist in one process, it does not share any variables and the only 
communication channel is a message queue. Serpent library provides two
builtin functions \texttt{thread\_send} and \texttt{thread\_receive} 
for message passing. Both \texttt{thread\_send} and \texttt{thread\_receive}
require \texttt{thread\_id} argument, which represent the caller thread Id. 
Note that the thread 2 need to define a \texttt{portime\_callback} function, 
in order to periodically check the message queue. Table 5.5 illustrate how 
two threads send and receive request via \texttt{thread\_send} and 
\texttt{thread\_receive} functions.
\begin{table}[htdp]
\centering
\begin{lstlisting}
//thread 1 send request 
Class PlayBtn(button)
  def init()
    super.init()
    this.method = 'send_play'

  def send_play(obj, event, x, y)
    thread_send(thread_id, "play;void")

------- separte line -------

//thread 2 receive request 
def porttime_callback(ms)
  cmd = thread_receive(thread_id)
  engine.process_cmd(cmd) //process command string
\end{lstlisting}
\caption[Push \& Pull Pattern]{Push \& Pull Pattern}
\end{table}

\subsection{Player Engine}
Table 5.6 is a pseudocode snippet of player engine, its main job is to read from  
GUI component through a message queue, parse the command string(the  
command string structure is defined chapter 3), and then look up a state-method 
matrix(refer to table 3.2 and 3.3) fetch the method to transit from current 
state to new state. Note that this segment of code snippet is also embeded 
within the \texttt{portime\_callback} function.


\begin{table}[htdp]
\centering
\begin{lstlisting}

def player_engine {
  cmd_str = read_from_GUI() //Player engine read request from GUI 
  cmds = parse_cmd_str(cmd_str) // parse command string into an array
  method = cmds[0]
  parameters = cmds[1]
  apply_function(method, parameters)
  cur_state = update_state(cur_state, method)
}

\end{lstlisting}
\caption[Player Engine Pseudocode]{Player Engine Pseudocode}
\end{table}

\section{Network Implementation}

\subsection{Initiate Connection}
Table 5.7 is a pseudocode snippet that illustrate how HCMP Player and   
Conductor build connection. Basically, it is a remote procedure model.
With Conductor in client side and HCMP Player in server side, HCMP Player  
initiate the connection and wait for reply from Conductor. The underlying
connection related function is provided inside zeromq library.

\begin{table}[htdp]
\centering
\begin{lstlisting}

// client side
def server {
  srv = socket (context, ZMQ_REP);
  srv.bind ("tcp://*:5555");
  srv.send(con_request);
}

------- separate line -------

// client side
def client {
  cli = socket (context, ZMQ_REP);
  cli.bind ("tcp://*:5555");
  con_req_str = cli.recv();
  reply = engine.process_req(con_req_str);
  cli.send(reply);
}

\end{lstlisting}
\caption[Initiate Connection Request]{Initiate Connection Request}
\end{table}

\subsection{Maintain Connection}  
Table 5.8 is the pseudocode snippet of observer pattern, 
which is used by HCMP Player and remote server for network
communication. 
The code has two part, subject and dependent, in real 
implementation, conductor is subject and HCMP Player 
is dependent. To begin communication 
, subject need to bind to certain port of local machine first, once 
a dependent register into the list, the subject begin to ``publish'' 
messages. In this situation, dependent just wait for comming 
messages and handle it.

\begin{table}[htdp]
\centering
\begin{lstlisting}
// subject thread 
def subject {
  context = zmq_ctx_new ()
	rc = zmq_bind (subject, "tcp://*:5556");
	rc = zmq_bind (subject, "ipc://HCMP_Player.ipc");

  while (1) {
    get_data_from_conductor();
    s_send (subject, update);
  }
  zmq_close ();
}

// dependent thread 
def dependent {
  context = zmq_ctx_new ();
  subscriber = zmq_socket (context, ZMQ_SUB);
  rc = zmq_connect (dependent, "tcp://localhost:5556");

  rc = zmq_setsockopt (dependent, ZMQ_SUBSCRIBE)
  cmd_string = s_recv (dependent);
  send_player_engine('cmd_str')

  zmq_close ();
}

\end{lstlisting}
\caption[Observer Pattern Pseudocode]{Observer Pattern Pseudocode}
\end{table}
