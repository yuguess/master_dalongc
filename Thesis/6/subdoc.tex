% this file is called up by thesis.tex
% content in this file will be fed into the main document

%: ----------------------- name of chapter  -------------------------
\chapter{Implementation} % top level followed by section, subsection


%: ----------------------- paths to graphics ------------------------

% change according to folder and file names
\ifpdf
    \graphicspath{{X/figures/PNG/}{X/figures/PDF/}{X/figures/}}
\else
    \graphicspath{{X/figures/EPS/}{X/figures/}}
\fi

\section{Overview}
Chapter 5 mainly describes the implemenation details of the HCMP Player. Some
pseudocode snippets are given to illustrate the relationship between 
different components. The focus will be on discussing some implementation 
related issues. 

\section{Development Environment}

The HCMP Player is developed using the scripting language Serpent\cite{Serpent}. 
Serpent is similar to Python, which has a simple, 
minimal syntax, dynamic typing and support for object-oriented programming. 
The best part of Serpent is 
that it has many convenient built-in functions for MIDI message processing, 
which greatly facilitates the development of music-related applications. 
As for the GUI development, the wxWidgets \cite{wxWidget} library provides one 
the of best solutions, and some core functions and objects have been 
integrated into the Serpent library. The Zeromq library is used to develop 
network related functions, it integrates with common network commmunication 
functions. The whole project is developed in the Linux 32-bit environment. 
Serpent provide a cross-platform solution, so the HCMP Player should also 
run on Windows or Mac platforms.

\section{GUI Implementation}
Table 5.1 illustrates the general code structure of the GUI component. 
It first parses the configuration file. If there is no configuration file 
in the current working directory, then the HCMP Player will create a new one.
The configuration file is stored in the current working directory with 
the name , ``\texttt{.player\_config}'' and contains all the HCMP Player  
related settings. For example, the default Conductor address and 
port number, recently opened file path, etc. The configuration file is written 
with plain ASCII text, user could easily modify the content. GUI 
component has 2 operations, \texttt{new} and \texttt{save} to create a new 
configuration file and save current configurations to the file.

After the configuration parsing and initialization job is complete, the 
the GUI begins to create each subcomponent. Each subcomponent has defined its 
own ``constructor'' function. Note that to make data accessing easier 
between different components, we pass in ``this'' (object itself) to each  
subcomponent.
\begin{table}[htdp]
\centering
\begin{lstlisting}

def GUI {
  data_init() //parse configuration file
  create_menu_panel(this) // initialize all the sub-components
  create_track_panel(this)
  create_data_panel(this)
  create_library_panel(this)
}

\end{lstlisting}
\caption[GUI Pseudocode Snippet]{GUI Pseudocode Snippet}
\end{table}

\subsection{Binding Method}
Table 5.2 is a Serpent code snippet which shows a simple way for the GUI object 
to bind with a method. We bind the Play button to a method, which sends a play 
request to the player engine. Note that the argument of the binding method is 
a string with the method name and parameters
separated by character \texttt{`;'}. The binding function's type is predefined 
inside Serpent, which has 4 predefined arguments. The first argument is the object 
itself. The second arugment is the event type. The third and fourth arguments are 
event position.
Not all of the event arguments are valid, since some other events provide 4 arguments, 
while some events may only need 2 arguments. In Serpent, all of the object events 
are dispatched through the same user defined fucntion, which gives the user full 
power to develop their own customized handlers.

\begin{table}[htdp]
\centering
\begin{lstlisting}

Class PlayBtn(button)
  def init()
    super.init()
    this.method = 'send_play'

  def send_play(obj, event, x, y)
    send_player_engine("play;void")

\end{lstlisting}
\caption[Binding Method]{Binding Method}
\end{table}

\subsection{Drawing Canvas}
 
Canvas objects define an area to draw images. In Serpent, the canvas 
object use the wxWidget canvas object. The canvas object has a full-size bitmap.
The drawing method first draws on the bitmap, then the whole bitmap is copied 
to the canvas object. This is a common technique to avoid flicker when a canvas is 
redraw In Serpent, a class needs to fulfill two conditions in order 
to be drawable. The first condition is
the class needs to be a subclass of the Canvas class. The second condition is that 
the class need to define its own paint method. Table 5.3 illustrate the user 
defined \texttt{Keyboard} class, which draws a keyboard on the canvas.

In the HCMP Player, there are two subclasses of the Canvas class, 
\texttt{Keyboard} and \texttt{MidiNotes}. The former draws a 108-key 
keyboard; the latter draws all MIDI notes on the canvas. Each class has a buffer 
to store all of the objects needed to be drawn. The play engine periodically 
sends ``synchronization'' signals. Everytime the GUI receives the synchronization 
signal, it will redraw the whole area.

\begin{table}[htdp]
\centering

\begin{lstlisting}
class Keyboard(Canvas)
    def init(parent, x, y, w, h, b, c)
      super.init(parent, x, y, w, h)

    def paint(x)
      draw_octave()
\end{lstlisting}
\caption[Serpent Canvas]{Serpent Canvas}
\end{table}

\section{Player Engine Implementation}

\subsection{Create a New Thread}
The Player engine runs inside a separate thread space. When the GUI thread completes 
all of the subcomponets' initialization work, it will create a new thread. The newly
spawned thread will execute the player engine code. In Table 5.4, we use the   
\texttt{thread\_create} built-in function of the Serpent standard library to create 
a new thread. It has 3 arguments. 
The first argument sets a period by which newly spawned thread will wake up and call 
a callback-style function. The mechanism can be used to set a message channel between 
two threads for communication purposes. The second argument specifies 
which script file to execute for the new thread. The third argument specifies the 
memory size for the new thread.

\begin{table}[htdp]
\centering

\begin{lstlisting}

g_object = GUI() //complete constructing GUI object
thread_create(2, "player_engine.srp", 0) // create new thread

\end{lstlisting}

\caption[Create New Thread]{Create New Thread}
\end{table}

\subsection{Thread Communication}

Serpent has a very primitive multi-thread interface. The thread concept is 
different from how threads work in the C programming language. Though all Serpent 
threads exist in one process, threads do not share any variables and the only 
communication channel is a message queue. The Serpent library provides two
built-in functions: \texttt{thread\_send} and \texttt{thread\_receive} 
for message passing. Both \texttt{thread\_send} and \texttt{thread\_receive}
require a \texttt{thread\_id} argument, which represents the caller thread ID. 
Note that the thread needs to define a \texttt{portime\_callback} function
in order to periodically check the message queue. Table 5.5 illustrates how 
two threads send and receive requests via the \texttt{thread\_send} and 
\texttt{thread\_receive} functions.
\begin{table}[htdp]
\centering
\begin{lstlisting}
//thread 1 send request 
Class PlayBtn(button)
  def init()
    super.init()
    this.method = 'send_play'

  def send_play(obj, event, x, y)
    thread_send(thread_id, "play;void")

------- separte line -------

//thread 2 receive request 
def porttime_callback(ms)
  cmd = thread_receive(thread_id)
  engine.process_cmd(cmd) //process command string
\end{lstlisting}
\caption[Push \& Pull Pattern]{Push \& Pull Pattern}
\end{table}

\subsection{Player Engine}
Table 5.6 is a pseudocode snippet of the player engine. Its main job is to read 
from GUI component through a message queue, parse the command string (the  
command string structure is defined in the Chapter 3), and then look up a state-method 
matrix (refer to Table 3.2 and 3.3) to fetch the method to transit from the 
current state to the new state. Note that this segment of code snippet is also 
embeded within the \texttt{portime\_callback} function.


\begin{table}[htdp]
\centering
\begin{lstlisting}

def player_engine {
  cmd_str = read_from_GUI() //Player engine read request from GUI 
  cmds = parse_cmd_str(cmd_str) // parse command string into an array
  method = cmds[0]
  parameters = cmds[1]
  apply_function(method, parameters)
  cur_state = update_state(cur_state, method)
}

\end{lstlisting}
\caption[Player Engine Pseudocode]{Player Engine Pseudocode}
\end{table}

\section{Network Implementation}

\subsection{Initiate Connection}
Table 5.7 is a pseudocode snippet that illustrates how the HCMP Player and   
Conductor build a connection. It is a remote procedure model.
With Conductor on the server side and the HCMP Player on the client side, HCMP 
Player initiates the connection and waits for a reply from Conductor. The 
underlying connection related function is provided inside the Zeromq library.

\begin{table}[htdp]
\centering
\begin{lstlisting}

// client side
def client {
  cli = socket (context, ZMQ_REP);
  cli.bind ("tcp://*:5555");
  cli.send(con_request);
}

------- separate line -------

// server side
def server {
  srv= socket (context, ZMQ_REP);
  srv.bind ("tcp://*:5555");
  con_req_str = srv.recv();
  reply = engine.process_req(con_req_str);
  srv.send(reply);
}

\end{lstlisting}
\caption[Initiate Connection Request]{Initiate Connection Request}
\end{table}

\subsection{Maintain Connection}  
Table 5.8 is the pseudocode snippet of the observer pattern, 
which is used by the HCMP Player and the Conductor for network
communication. The code has two parts, {\bf Subject} and {\bf Dependent}. In 
the real implementation, the Conductor is the {\bf Subject} and the HCMP Player 
is the {\bf Dependent}. To begin communication, the {\bf Subject} needs to 
bind to a certain port of the local machine first. 
Once the {\bf Dependent} registers into the list, the {\bf Subject} begins 
to ``publish'' messages. In this situation, the {\bf Dependent} just waits 
for coming messages and handles them.

\begin{table}[htdp]
\centering
\begin{lstlisting}
// subject thread 
def subject {
  context = zmq_ctx_new ()
	rc = zmq_bind (subject, "tcp://*:5556");
	rc = zmq_bind (subject, "ipc://HCMP_Player.ipc");

  while (1) {
    get_data_from_conductor();
    s_send (subject, update);
  }
  zmq_close ();
}

// dependent thread 
def dependent {
  context = zmq_ctx_new ();
  subscriber = zmq_socket (context, ZMQ_SUB);
  rc = zmq_connect (dependent, "tcp://localhost:5556");

  rc = zmq_setsockopt (dependent, ZMQ_SUBSCRIBE)
  cmd_string = s_recv (dependent);
  send_player_engine("cmd_str")

  zmq_close ();
}

\end{lstlisting}
\caption[Observer Pattern Pseudocode]{Observer Pattern Pseudocode}
\end{table}
