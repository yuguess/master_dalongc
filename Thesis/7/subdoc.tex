% this file is called up by thesis.tex
% content in this file will be fed into the main document

%: ----------------------- name of chapter  -------------------------
\chapter{Evaluation} % top level followed by section, subsection


%: ----------------------- paths to graphics ------------------------

% change according to folder and file names
\ifpdf
    \graphicspath{{X/figures/PNG/}{X/figures/PDF/}{X/figures/}}
\else
    \graphicspath{{X/figures/EPS/}{X/figures/}}
\fi

%: ----------------------- contents from here ------------------------

\section{Overview}
Chapter 6 mainly focus on evaluation of the project. In 
this chapter, we define the successful criteria of the project and discuss
how to meet those requirements. A complete list of required features is
listed. At last part of chapter, we also make brief performance test and 
discuss some potential performance issues.

\section{Functional Evaluation}
Based on the project design from previous chapters, the evaluation of the 
project mainly contains 
two parts. To reach the successful criteria, the project not only need to 
satisfy the functional requirement but also all of performance requirements. 
The 
functional requirement cover all the basic function of the HCMP Player, 
each functional requirement will be 
either manully tested or automatically tested by scripts. As for the
performance test, some code segment will be added to original program to
test the general performance issues.

\subsection{Stand-alone Mode Requirement}
\begin{enumerate}
  \item User should be able to load and play normal midi files.
  \item User should be able to play, pause, stop while playing midi file.  
  \item User should be able to freely adjust tempo, while plying the midi file.   
  \item Data visualization component should correctly show the midi note 
        and synchronize with audio stream while playing the midi file.
  \item Each track component should match to color on data visualization component.
  \item Each track component should be able to mute, solo and adjust volume.  
  \item The player should contain a simple midi library to manage midi files. 
        The library should automatically remember recently imported midi files.
  \item The player should provide a complete configuraion interface for the 
        user, all user's setting should be saved automatically.
  \item The configuration of HCMP Player should be saved into a file.
\end{enumerate}

\subsection{Network Mode Requirement}
\begin{enumerate}
  \item Each instance of HCMP Player runs in its own process space, one player 
        should not effect other player's function or performance. 
        
  \item Each HCMP Plyaer should be launched separately by user. After 
        launching, the midi player should automatically connect to a conductor 
        based on its configuration.

  \item If HCMP Player failed to establish connection with remote server, it should  
        be rollback to stand-alone mode without crashing.
  \item The Conductor communicate with midi player through a predefined API, 
        the API is defined in Chpater 5. After establishing connection with remote   
        server, HCMP Player should register into Conductor's dependent list.   
        Once received command, the Conductor send message to all HCMP Player that    
        previously registered in the dependent list. 

  \item Multiple instances of midi player should be able to connect to one conductor. 
  \item After establishing connection with conductor, the midi player 
        should be able to receive tempo informaiton from conductor and 
        synchronize with it.     
  \item The conductor should be able to explicitly command certain midi player to 
        load specified configuration file or it could directly 
        send configuration information through network to that midi player.
  \item 
    The HCMP Player is compatible with ZeroMQ \cite{zeromq} api, 
    and 4 core functions will be like
    \begin{itemize}
      \item play - ask a player to start playing 
      \item stop - stop a player from playing 
      \item update time-map - send new time-map to the player 
      \item set position - set position to certain beat 
    \end{itemize}
\end{enumerate}

\subsection{Performance Requirement}
\begin{enumerate}
  \item The HCMP Player GUI part should always response to user input without 
        significant delay, and the GUI thread should not hang there.
  \item The HCMP Player should response to conductor without significant delay.
  \item The HCMP Player should use no more than 100MB memory.
  \item The HCMP Player should use reasonable amount of CPU while playing midi file.
  \item Hopefully, the midi player shoud not crash without any reason. 
\end{enumerate}

\section{Test Plan}
\subsection{Functional Test}
Functional test aims to test all the features list above, some of key test 
cases are list below.
\begin{itemize}
  \item Test case 1 -- load 10 different midi files, each has different length, 
        including a ill midi file, play, pause and stop. 
  \item Test case 2 -- load 10 different midi files, change tempo from high to 
        low, and then from low back to high, change the tempo every 0.5 second.
  \item Test case 3 -- load 10 different midi files, use track panel to mute
        from first track to last track, then do the similar action with volume
        adjust.
  \item Test case 4 -- load 10 different midi files in HCMP Player library, click
        each track to import, then delete all the files in disk, and clikc each 
        track again.  
  \item Test case 5 -- create a virtual conductor and create a HCMP Player use  
        network function to connect to conductor. Conductor issue command
        to test each network API and make sure its work properly. 
  \item Test case 6 -- create a virtual conductor and create 10 HCMP Players. 
        All connected to conductor. Conductor issue command to test each network
        API and make sure its work properly.
  \item Test case 7 -- create a virtual conductor and create a HCMP Player to connect
        to conductor. While HCMP Player playing the file, cut the network connection,
        and HCMP Player behave properly.
  \item Test case 8 -- create a virtual conductor and create 10 HCMP Players, all
        connect to conductor. Conductor explicitly ask each HCMP Player to play with
        certian configuration file.
\end{itemize}

\subsection{Perfermance Test}
Performance test aims to test the general performance of HCMP Player. Two important  
aspects for performance test are canvas redraw test and midi event test. 

As described in previous chapters, canvas redraw the whole bitmap everytime 
the beat update, the bitmap will be updated approximately 20 times per second.
Even though the bitmap size is relatively small, it still possible to become a bottelneck
for the whole system, any inefficiency in the code will lead to GUI lag or hang there
without responding to user input. The test case 1 try to measure the general 
accumlative time spent on redrawing. 

Another issue is about midi event, the player engine use real-time scheduler to  
contantly poll and execute event, ideally, the event is dispatched at its 
schedule time, or at least we should control so that event dispatched within 
certain limited time, the test case 2 aims to measure the overall latency between 
ideal schedule time and actual dispatch time. 
\subsubsection{Performance Test Case}
\begin{itemize}
  \item Test case 1 -- measure the time between canvas redraw segment code, and add 
        each redraw time together to gain total time spend on drawing, calculate the 
        ratio of redraw time and total time.
  \item Test case 2 -- In player engine, after the event has been dispatched, 
        check the ideal time stored in that event, and get current time, calculate 
        the latency. change the tempo, and measure the latency again.
\end{itemize}
