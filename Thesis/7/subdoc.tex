% this file is called up by thesis.tex
% content in this file will be fed into the main document

%: ----------------------- name of chapter  -------------------------
\chapter{Evaluation} % top level followed by section, subsection


%: ----------------------- paths to graphics ------------------------

% change according to folder and file names
\ifpdf
    \graphicspath{{X/figures/PNG/}{X/figures/PDF/}{X/figures/}}
\else
    \graphicspath{{X/figures/EPS/}{X/figures/}}
\fi

%: ----------------------- contents from here ------------------------

\section{Overview}
Chapter 6 mainly focuses on the evaluation of the project. In
this chapter, we define the successful criteria of the project and discuss
how to meet those requirements. A complete list of required features is
listed. At the last part of the chapter, we also make a brief performance test 
and discuss some potential performance issues.

\section{Functional Evaluation}
Based on the project design from previous chapters, the evaluation of the 
project contains two parts. To reach the successful criteria, the project not 
only need to satisfy the functional requirements but also all of the performance 
requirements. The 
functional requirement cover all the basic function of the HCMP Player.
Each functional requirement will be 
either manully tested or automatically tested by scripts. As for the
performance test, some code segments will be added to the original program to
test the general performance issues.

\subsection{Stand-alone Mode Requirement}
\begin{enumerate}
  \item The user should be able to load and play normal MIDI files.
  \item The user should be able to play, pause, and stop while playing MIDI files.  
  \item The user should be able to freely adjust tempo, while playing the MIDI files.
  \item Data visualization components should correctly show the MIDI notes 
        and synchronize with audio stream while playing the MIDI files.
  \item Each track component should match to colors on the data visualization 
        component.
  \item Each track component should be able to mute, solo and adjust volume.  
  \item The player should contain a simple MIDI library to manage MIDI files. 
        The library should automatically remember recently imported MIDI files.
  \item The player should provide a complete configuration interface for the 
        user. All of user's setting should be saved automatically.
  \item The configuration of HCMP Player should be saved into a file.
\end{enumerate}

\subsection{Network Mode Requirement}
\begin{enumerate}
  \item Each instance of HCMP Player runs in its own process space. One player 
        should not effect other player's function or performance. 
        
  \item Each HCMP Plyaer should be launched separately by the user. After 
        launching, the MIDI player should automatically connect to a Conductor 
        based on its configuration.

  \item If HCMP Player failed to establish a connection with the remote server, 
        it should rollback to the stand-alone mode without crashing.

  \item The Conductor communicates with MIDI player through a predefined API. 
        The API is defined in Chapter 5. After establishing a connection with 
        the remote server, HCMP Player should register into Conductor's 
        {\bf Dependent} list. Once a command is received, the Conductor sends 
        messages to all the HCMP Players that previously registered in the 
        {\bf Dependent} list. 

  \item Multiple instances of the HCMP Player should be able to connect to one Conductor. 
  \item After establishing a connection with Conductor, the MIDI player 
        should be able to receive tempo information from Conductor and 
        synchronize with it.     
  \item The Conductor should be able to explicitly command certain MIDI players to 
        load a specified configuration file or it could directly 
        send configuration information through the network to that MIDI player.
  \item 
    The HCMP Player is compatible with Zeromq\cite{zeromq} API, 
    and the 4 core functions will be
    \begin{itemize}
      \item \texttt{play} - ask a player to start playing 
      \item \texttt{stop} - stop a player from playing 
      \item \texttt{update time-map} - send a new time-map to the player 
      \item \texttt{set position} - set the position to a certain beat 
    \end{itemize}
\end{enumerate}

\subsection{Performance Requirement}
\begin{enumerate}
  \item The HCMP Player GUI should always respond to user input without 
        significant delay, and the GUI thread should not hang.
  \item The HCMP Player should respond to Conductor without significant delay.
  \item The HCMP Player should use no more than 100MB of memory.
  \item The HCMP Player should use a reasonable amount of CPU while playing 
        MIDI files.
  \item The midi player should not crash.
\end{enumerate}

\section{Test Plan}
\subsection{Functional Test}
A functional test aims to test all of the features listed above. Some of the 
key test cases are listed below.
\begin{itemize}
  \item Test case 1 -- Load 10 different MIDI files, each having a different
        length, including a invalid MIDI file, play, pause and stop. 
  \item Test case 2 -- Load 10 different MIDI files, change the tempo from high to 
        low, and then from low back to high. Change the tempo every 0.5s.
  \item Test case 3 -- Load 10 different midi files, use the track panel to mute
        from the first track to the last track, then do a similar action with the 
        volume adjust.
  \item Test case 4 -- Load 10 different midi files into HCMP Player library. Click
        each track to import, then delete all of the files on the disk, and 
        click each track again.  
  \item Test case 5 -- Create a virtual conductor and create a HCMP Player to use  
        of network function to connect to Conductor. Conductor issues a command
        to test each network API to make sure works properly. 
  \item Test case 6 -- Create a virtual conductor and create 10 HCMP Players, 
        all connected to Conductor. Conductor issues a command to test each network
        API to make sure it works properly.
  \item Test case 7 -- Create a virtual conductor and create a HCMP Player to connect
        to Conductor. While HCMP Player is playing the file, cut the network connection
        and determine if HCMP Player behaves properly.
  \item Test case 8 -- Create a virtual conductor and create 10 HCMP Players, observer
        if all connected to Conductor. Conductor explicitly asks each HCMP Player to 
        play with a certain configuration file.
\end{itemize}

\subsection{Perfermance Test}
The performance test aims to test the general performance of the HCMP Player. Two 
important aspects for the performance test are the canvas redraw test and the MIDI
event test. 

As described in previous chapters, the canvas redraws the whole bitmap everytime 
the beat updates. The bitmap will be updated approximately 20 times per second.
Even though the bitmap size is relatively small, it is still possible to become 
a bottelneck for the whole system. Any inefficiency in the code will lead to GUI 
lag or it may hang there without responding to user input. The test case 1 tries 
to measure the general accumlative time spent on redrawing. 

Another issue involves MIDI events. The player engine uses real-time scheduler to  
constantly poll and execute events. Ideally, the event is dispatched at its 
scheduled time, or at least we should control it so that events are dispatched within 
a certain limited time. The test case 2 aims to measure the overall latency between 
the ideal scheduled time and the actual dispatch time. 
\subsubsection{Performance Test Case}
\begin{itemize}
  \item Test case 1 -- Measure the time between the canvas redraw rontinue, and 
        add each redraw time together to obtain total time spent on drawing. 
        Calculate the ratio of the redraw time and the total time.
  \item Test case 2 -- In the player engine, after the event has been dispatched, 
        check the ideal time stored in that event, get the current time, calculate 
        the latency. Change the tempos, and measure the latency again.
\end{itemize}
